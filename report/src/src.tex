\section{Веб-сервер}
В качестве веб-сервера использовался Nginx, настроенный на стандартный порт 80. В конфигурационном файле определено три location:
\begin{enumerate} 
  \item \textbf{/} - проксирует запросы на сервер приложений.
  \item \textbf{/protected/} - для внутренних (internal) запросов. Обслуживает приватные файлы пользователя. Сервер приложений переадресует запрос на этот location используя поле \textbf{X-Accel-Redirect} в заголовке.
  \item \textbf{/static/} - раздает статические файлы.
\end{enumerate}
\section{Сервер-приложений}
Сервер приложений реализован с помощью фреймворка Django и настроен на порт 8000. Реализовано полноценное \textbf{RESTful} \textbf{API} приложения магазина книг. Для представления данных в запросе/ответе используется \textbf{JSON}. Для для сериализации/десериализации используется пакет \textbf{djangorestframework}.
\section{База данных и ORM}
В приложении использовалась база PostgreSQL \cite{postgresql_documentation}. Среди её достоинств можно выделить высокий уровень соответствия с стандартом SQL, чего нельзя, например, сказать об Oracle. Открытый исходный код.

Были разработаны модели представляющие \textit{книгу}, \textit{автора}, \textit{жанр} и \textit{пользователя системы}. Затем для данных моделей была написана реализация с использование класса \textbf{django.db.models.Model}.
\section{S3-хранилища}
Было использовано Mail.ru Cloud Solutions (MCS) хранилище. Предусмотренна возможность загрузки и скачивания пользовательского изображения (<<аватарки>>). Предполагается, что это изображение доступно только пользователю загрузившему его, а также пользователю с повышенными привилегиями (admin). Все остальные при попытке обратиться к данному URL получат 403 Forbidden. Анонимные пользователи будут переадресованы на страницу авторизации. В зависимости от значения поля \textbf{DEFAULT\_FILE\_STORAGE} конфигурации для хранения <<аватарок>> будет использованы либо локальная файловая система, либо облачное хранилище.
\section{Авторизация}
Для авторизации использовались встроенные сре
\section{Тестирование}
Для тестирования \textbf{REST} \textbf{API} сервера приложения использовались встроенные средства Django: \textbf{django.test.TestCase} для прогона тестовых случаев и \textbf{django.test.Client} для эмуляции клиента.
Чтобы протестировать код отвечающий за сохранение <<аватарки>> был сымитирован метод \textbf{save} файлового хранилища. Для имитации использовался метод \textbf{patch} из пакета \textbf{unittest.mock}. Была протестирована авторизация пользователя в браузере Firefox. Для этого использовался фреймворк \textbf{selenium}.
\section{Исходный код}
Ознакомится с исходным кодом приложения можно на GitHub \cite{project_source}.
\pagebreak
